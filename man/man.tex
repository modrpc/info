\documentclass{myproc}
\usepackage{mydef,myenv}
\usepackage{kotex}
\usepackage{mathptm}
\usepackage{MinionPro,MnSymbol}

%\newcommand{\arc}[1]{\mbox{$\langle{\textit{#1}}\rangle$}}


%\def\sbf{\sf\bfseries}
\def\sbf{\bfseries}
\pagestyle{empty}

\begin{document}
\small

\begin{center}
  \textcolor{blue2}{\large\bf 매장 운영 매뉴얼}
\end{center}

\section{작업}
\subsection{오픈}
\subsection{마감}
\subsubsection{쉐이크기계 청소}
\bit
\w 이틀에 한번 청소할 것
\w 청소 방법:
  \ben
  \w 수도를 틀어 물을 채운 후, 레버를 왼쪽으로 돌린다 (청소모드).
  \w 빈 쉐이크 통에 물을 받아내서 싱크대에 버린다. 
  \w 위의 1, 2번 작업을 2번 더 반복한다.
  \w 쉐이크 기계를 분해해서 싱크대에서 세척한다. 세척은 찬물을 사용한다.
  \w 쉐이크 기계를 조립한다. 필요한 곳에 식용 그리스를 바른다.
  \w 수도를 틀어 물을 채운 후, 소독약을 작은 종이컵 반정도 분량을 추가하고,
  레버를 왼쪽으로 돌린다.
  \w 빈 쉬이크 통에 물을 받아내서 싱크대에 버린다.
  \w 위의 6, 7번 작업을 2번 더 반복한다.
  \een
\eit


\tableofcontents
\end{document}


% LocalWords:  Cheoljoo Jeong vertices endvertices endvertex cutvertex bc algo
% LocalWords:  cutvertices iff uv cocycles coboundary
