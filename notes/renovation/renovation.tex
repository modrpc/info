\documentclass{myproc}
\usepackage{kotex,pgfplots}
\usepackage{mathptm,mydef}
\usepackage{alltt}
\usepackage[T1]{fontenc}
\usepackage[all,knot]{xy}
\usepackage{MinionPro}
\usepackage{times}

\usepackage{pgfplots}
\usepackage{tikz}

\usepackage{hyperref}
\hypersetup{
  colorlinks,
  citecolor=black,
  filecolor=black,
  linkcolor=blue,
  urlcolor=black
}

\def\sbf{\sf\bfseries}


\begin{document}

\begin{center}
  \textcolor{blue2}{\large\bf Remodeling Requirements}
  \\
  \vspace*{0.8cm}

\end{center}

\small

\section{코롬방제과점}
\subsection{기본}
\subsubsection{도면 작성}

\subsubsection{수도배관공사}
\bit
\w 오래된 배관 교체할 것
\eit

\subsubsection{전기공사}
\bit
\w \textcolor{red2}{\bf{}계약전력 증설}
\w 전기 배전함 정리
\w 간판/내부 전등 등을 일목요연하게 정리할 것; SORT!!!
\eit

\subsection{외관/입구}
\bit
\w 외관은 최대한 원형을 유지할 것
\w 출입문 수리 또는 교체 (자동문 교체?)
\eit

\subsection{골목}
\bit
\w 바닥공사 다시 할것
\w 배수가 원활히 되도록 공사
\w 에어컨 실외기 정리: 전부 옥상으로 올릴 것
\w 렉산으로 된 곳을 새로 설치
\w \textcolor{red2}{\bf{}골목창고}: 입구를 막아버리고, 코롬방1층 내부 창고와
연결하거나 깨끗하게 수리해서 골목쪽에서 사용
\eit


\subsection{코롬방제과점 1층}
\subsubsection{전기 outlet}
\bit
\w \textcolor{red2}{power outlet을 적절한 위치에, 충분히 만들어놓을 것}
\w 입구, 기둥, 벽, 카운터 뒷쪽 벽 등 
\eit

\subsubsection{사무실}
\bit
\w 책상 1개, 의자 2개
\w 컴퓨터, 프린터, 서류함
\eit

\subsubsection{비품창고}
\bit
\w 선반을 설치해서 (2층에 있는 선반 이동),
     추가 저장공간 마련 (상자류, 음료류, 기타 비품/집기)
\w 작은 couch (휴식용)
\eit

\subsubsection{기존 창고}
\bit
\w 벽을 없애서 확장할 것
\w 재료: 음료수, 잼 등
\w 비품: 고무장갑, 비닐장갑, 키친타월 등 
\eit

\subsubsection{천장}
\bit
\w 노출 후, 나무 패널(?)
\eit

\subsubsection{환기/냉난방 설비}
\bit
\w \textcolor{red2}{냉난방기는 천정형으로 할 것}: 빵 진열대에서 빗겨난 곳에 설치
\w \bb{환기}: 환기 설비; 앞문을 열지 않은 상태에서 배기가 가능하도록
\eit

\subsubsection{조명}
\bit
\w \textcolor{red2}{\bf{}밝게!!!}
\w 중요한 위치는 B\"{A}RO 램프 설치
\w 나머지는 국산 ``수'' 램프 설치
\w 천장 조명을 특색있게 (디자인한 형태의 전등)
\eit

\subsubsection{진열대}
\bit
\w 진열대는 ``age well over time''할만한 소재를 사용할것 -- 고가의 재료를
사용할 필요는 없음
\w 적어도 다음의 섹션이 구분되어야 함
  \bit
  \w 일반단과자빵  (단팥빵, 크림빵, 소보루 등)
  \w 건강빵 (바게트, 치아바타, 캄파뉴 등)
  \w 인기상품 (크림치즈 바게트, 목포 주전부리)
  \w 파이류 (크로아상, 뺑오쇼콜라 등)
  \w 튀김류 
  \eit
\w 덮개가 필요한 제품의 경우, 덮개 설치
\w 2단, 3단 으로 공간 절약
\w 빵을 나무 트레이에 놓을 것인지, 아니면 빵판 그대로 놓을지 결정
\w 건강빵의 경우, Passon 5처럼 바로 놓을 수 있도록 판을 짜놓을 것
\w \textcolor{red2}{\bf{}기둥을 $360^\circ$로 둘러싼 형태의 2/3단 진열대 (기둥을
  원통형으로 바꿀 수 있는지?)}
\eit

\subsubsection{전기}
\bit
\w 배전판 등 전기 관련 설비는 밖에서 보이지 않는 공간에 설치
\eit

\subsubsection{쇼케이스}
\bit
\w 생일케익용 쇼케이스
\w 카스텔라, 롤케익용 쇼케이스
\w Cold 샌드위치 용 쇼케이스
\w 조각케익용 쇼케이스
\w 튀김용 \textcolor{red2}{온장 쇼케이스} 필요한지 확인
\eit

\subsubsection{카운터}
\bit
\w 카운터 소재: 천연 대리석 vs 인조 대리석
\w 카운터에 전화, 인터넷 등 선을 깨끗하게 정리
\w POS기는 연결선이 겉으로 드러나지 않도록
\w \textcolor{blue2}{La Colombe} 참고
\eit

\subsubsection{덤웨이터}
\bit
\w 1층/2층 간의 덤웨이터
\w 빵 rack 2,3 개가 동시에 들어갈 정도의 크기
\w 덤웨이터는 칸막이를 이용하여 방문객으로부터 차단하는 것이 좋은지  (방문객은
rack이 오가는 것만 보도록) 아니면 개방하는 것이 좋은지
\eit

\subsubsection{벽}
\bit
\w 벽 전체를 통일할지, 아니면 카운터 이후와 진열공간을 다르게 할지 결정
\w OPTION: La Colombe, Holmes Bakery 류의 하얀 타일  (카운터부터), 진열공간은
다른 재질
\eit

\subsubsection{바닥}
\bit
\w 바닥 전체를 통일할지, 아니면 카운터 이후와 진열공간을 다르게 할지 결정
\w 청소가 용이해야 하고, ``age well over time'' 또는 easily-replaceable or ?
\w OPTION: 진한 색 나무마루
\eit

\subsubsection{계단}
\bit
\w 계단 손잡이 바꿀 것
\w 계단 소재 변경
\eit

\subsubsection{제빵 작업공간}
\bit
\w 로터리 오븐 1대; 데크오븐 1대
\w 하부에 냉장고가 달린 작업대 (스키피오 냉장 테이블)
\w 하부에, 소스 저장
\w {\bf 쓰레기통은 크게}
\eit

\subsubsection{포장 작업공간}
\bit
\w 빵 포장, 택배 작업, 박싱 작업 등 
\w 작업대는 ``age well''하는 소재 (상판은 인조 대리석으로 하더라도); 예를
들어, 합판 재질은  보기에도 안 좋고 쉽게 망가짐
\w 작업대/작업공간은 청소가 용이해야 함
\w 작업공간 밑에 저장공간 확보 (문을 달아서 먼지, 오물이 쌓이지 안도록)
\w {\bf 쓰레기통은 크게}
\eit

\subsubsection{음료 작업공간}
\bit
\w 커피기계, 온수기, 그라인더
\w 빙수기기 1대, 밀크셰이크 기계 2대, 아이스크림기계 1대를 놓을 수 있는 공간
\w 싱크대를 옆에 위치 할 것 (faucet은 2개 이상);
아이스크림 기기 청소를 위해 호스 설치
\w \textcolor{red2}{하부에 제빙기 2대 설치}
\w \textcolor{red2}{식기 세척기와 식기 건조기 설치}
\w 컵, 기타 용기 저장을 위한 충분한 공간
\w {\bf 쓰레기통은 크게}
\eit

\subsection{냉장/냉동고}
\bit
\w 테이블 냉장고 \#1: 소스, 밀크셰이크 등
\w 냉장냉동고 \#2: 밀크셰이크재료, 카페재료
\w 냉동고 \#1: 각종 디저트, 다쿠아즈, 마카롱
\w 냉동고 \#3: 새우/마늘 바케트 (번팬용 냉동고)
\eit

\subsubsection{디지탈 사이니지}
\bit
\w 카운터 상단: 3개
\w 쇼케이스 근처:
\w 보 기둥: 바게트, 주전부리 광고
\eit

\subsubsection{기타}
\bit
\w 여름에 벌레 유입을 막을 수 있는 장치 
\eit

\subsection{코롬방제과점 2층}
\subsubsection{기본}
\bit
\w \textcolor{red2}{\bf{}2층이 기계의 총중량을 버틸 수 있는지 확인할 것}
\w \bb{용이한 청소}: 청소를 쉽게 할 수 있어야 함  (incl. 기계 밑에 쌓인 오물)
\w \bb{평평한 바닥}: 특별한 이득이 없는 한, 바닥을 기울어지지 않게 시공한다. 
\eit

\subsubsection{밀폐공간}
\bit
\w \textcolor{blue2}{\bf{}파이실}: 파이실 내 낮은 온도 유지 가능; 적당한 크기
\w 칸막이벽을 이용한 reconfigurable한 공간
\eit

\subsubsection{화장실}
\bit
\w 둘 다 없애거나 하나만 놔두고 없앨 것
\w 깨끗하게 renovation
\eit


\subsubsection{기계}
\bit
\w 기본적으로, \textcolor{red2}{\bf 제품 품질 향상}, \textcolor{red2}{\bf 인력
  절감}에 도움이 되는 기계 도입; 그리고 노후 장비 교체
\w 돌가마
\w 대리석 작업대
\w 냉장/냉동고 (창고 외)
\eit

\subsubsection{냉동/냉장설비}
\bit
\w walk-in 냉장고, walk-in 냉동고
\w 내부 청소가 쉽도록, 넓게 설계
\w 하부 받침대를 처음부터 설계
\eit

\subsection{코롬방제과점 3층}
\subsubsection{화장실}
\bit
\w 남녀 화장실 깨끗하게 수리
\eit
\subsubsection{라커룸}
\bit
\w 에어컨 설치
\w 깨끗하게 수리
\eit

\subsubsection{냉장/냉동창고}

\section{니콘안경}
\subsection{도면 작성}
\subsection{기본}
\bit
\w 코롬방제과점과 니콘안경간의 이동이 쉽도록
\w 니콘안경으로부터 코롬방 내부 상황이 보이도록 (벽을 유리로 대체?)
\eit

\subsection{니콘안경 1층}
\subsubsection{건물 외관}
\bit
\w 2층은 전면 유리
\w ``SINCE 1949 MOKPO'' 로고 사용할 것
\w 좋은 문구 ``목포는 1897년 개항, 일제시대 수탈기지, 해방, 산업화시대...... 애환을 가진
도시입니다.  이곳 오거리는 목포의 중심지였습니다. 광주항쟁, 민주화운동의
중심이었고 . 코롬방제과점은 이곳에서 70년이 넘는 세월동안 변치않고 목포시민과
함께 있습니다..". 목포의 중심지였습니다. 민주화항쟁의 거리였고, 또 전시회가
열리고, 문인, 예술가, 음악인, 사진가 유치환 선생이 ... ???가 시를 논하고, .... 하던 낭만과 문화의
거리였습니다.
\w 오거리 역사책 참고

\w ``https://110callcenter.tistory.com/2919''
\w 오거리 특별전 다시 열것!
\eit

\subsubsection{카운터/주방}
\bit
\w 니콘안경 1층 카운터와 코롬방제과점 1층 카운터를 line-up하여 다른 건물에
있지만, 일체감이 있는 느낌을 주도록
\w 카운터 옆에 작은 쇼케이스를 만들어서 조각케익등 진열
\w 음료/디저트  쇼케이스를 놓을 것
\w 집기 세척을 위한 공간 마련 (씽크대 포함)
\w \textcolor{red2}{\bf 핫 샌드위치 제조가 가능한 공간을 만들 것}
\eit


\subsubsection{디지탈 사이니지}
\bit
\w 크게 3개 연결  (메뉴, 마케팅, etc)
\eit

\subsubsection{서비스 테이블}
\bit
\w 분리수거 
\eit

\subsubsection{좌석}

\subsubsection{화장실}
\bit
\w 적어도 남녀 각기 2명 동시 사용가능한 크기
\w 고급스럽고 깨끗하게
\eit

\subsubsection{덤웨이터}
\bit
\w 1/2층간 집기류, hot sandwich등을 나를수 있는 덤웨이터
\eit

\subsection{니콘안경 2층}
\subsubsection{주방}
\bit
\w 코롬방제과점 2층과와 연결되는 문
\w 필요한 경우, hot sandwich의 중간 산출물을 코롬방제과점 2층으로부터 받아와서
완성 시킬 수 있는 시설
\eit

\subsubsection{좌석}
\bit
\w 유리창 근처에 앉을 수 있는 자리 마련
\eit

\subsubsection{서비스 테이블}




\vspace*{2cm}

\tableofcontents
\end{document}



