\documentclass{myproc}
\usepackage{kotex,pgfplots}
\usepackage{mathptm,mydef}
\usepackage{alltt}
\usepackage[T1]{fontenc}
\usepackage[all,knot]{xy}
\usepackage{MinionPro}
%\usepackage{times}

\usepackage{pgfplots}
\usepackage{tikz}

\usepackage{hyperref}
\hypersetup{
  colorlinks,
  citecolor=black,
  filecolor=black,
  linkcolor=blue,
  urlcolor=black
}

\def\sbf{\sf\bfseries}


\begin{document}

\begin{center}
  \textcolor{blue2}{\large\bf 리브랜딩/리모델링 계획}
  \\
  \vspace*{0.8cm}

\end{center}

\small

\section{브랜딩}

\section{디자인}

\section{시공}
\subsection{작업 순서}
\ben
\w 니콘안경 1,2층 공사 (니콘 안경 측 통로는 완성이 되어있어야 함)
\w 코롬방제과점 전기공사/수도배관공사
\w 코롬방제과점 2층 공사 (공사가 끝나면 건물간 2층 통로는 완성이 되어야 하고,
공장은 2층으로 옮겨야 함)
\w 코롬방제과점 1층 공사 (1층 공사 중, 니콘안경 건물에서 임시영업; 제품은
코롬방제과점/니콘안경 2층 통로를 통해서 이동)
\een

\subsection{코롬방제과점 공사}
\subsubsection{전기공사}
\bit
\w 계약전력 증설
\w 전기 배전판 정리
\eit
\subsubsection{수도배관공사}
\bit
\w 오래된 배관 교체할 것
\eit
\subsubsection{외관}
\bit
\w 미니멀하게 수정
\w 조명 수정
\eit
\subsubsection{1층 매장}
\bit
\w 1층 전체를  매장으로 수리
\w 현재 공장 자리에 1) 포장공간, 2) 음료/셰이크 제작공간, 3) 싱크대, 4) 포장지
저장공간, 5) 사무실, 6) 간단한 제빵 시설 마련
\w 1층 매장과 니콘 안경 1층 사이의  통로 만들것
\w 카운터 쪽과 매장 쪽 인테리어에 차이를 둘 것
\w 매장 쪽은 70년이 넘은 오래된 빵집의 분위기
\w 카운터 쪽은 좀 더 모던한 분위기
\eit

\subsubsection{2층 공장}
\bit
\w 2층 전체 공장
\w 냉동/냉장 창고 제작
\w 계단에서 화장실, 창가 좌석 쪽 통로는 손님이 accessible하게 할 것
\w 2층 공장과 니콘 안경 2층 사이의  통로 만들것
\eit

\subsubsection{골목 정리}


\subsection{니콘안경}
\subsubsection{외관}

\subsubsection{1층 매장/카페}
\bit
\w 카운터가 있어야함 (코롬방 1층 매장 공사때 임시영업)
\w 전체적으로 모던한 분위기
\eit

\subsubsection{2층 카페}
\bit
\w 최소한의 디자인; 모던한 분위기
\w 화장실
\eit



\section{디스플레이/데코}
\subsection{Digital Signage}
\subsection{포토존}


\section{패키징}
\subsection{기본빵 포장}
\subsection{페이스트리 포장}
\subsection{케익/디저트 포장}
\subsection{선물 포장}

\section{제품 라인업/개발}
\subsection{시그니처빵}
\bit
\w 바게트류, 목포주전부리3종, etc
\eit
\subsection{기본빵}
\bit
\w 단과자, 튀김, 샌드위치
\eit
\subsection{페이스트리}
\subsection{디저트/조각케익}
\subsection{케익류}

\section{마케팅/세일즈}
\subsection{SNS 마케팅}
\bit
\w 유화제/개량제를 사용하지 않음을 선전; 유화제/개량제를 넣으면 더 부드러움이
오래가지만, 건강에는 좋지 않음
\w 생크림: 저렴하지만, 첨가물이 들어간 휘핑크림 대신 100\% 유크림으로 만든
생크림을 사용 (그중 유지방 함량이 가장 높은 파스퇴르 생크림 사용)
\eit
\subsection{이벤트}
\subsection{통신판매/배달}

\vspace*{2cm}

\tableofcontents

\end{document}



