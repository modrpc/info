\documentclass{myproc}
\usepackage{kotex,pgfplots}
\usepackage{mathptm,mydef}
\usepackage{alltt}
\usepackage[T1]{fontenc}
\usepackage[all,knot]{xy}
\usepackage{MinionPro}
\usepackage{times}

\usepackage{pgfplots}
\usepackage{tikz}

\usepackage{hyperref}
\hypersetup{
  colorlinks,
  citecolor=black,
  filecolor=black,
  linkcolor=blue,
  urlcolor=black
}

\def\sbf{\sf\bfseries}


\begin{document}

\begin{center}
  \textcolor{blue2}{\large\bf 공장/매장 매뉴얼}
  \\
  \vspace*{0.8cm}

\end{center}

\small

\section{공장}
\subsection{성수기 생산량}
\bit
\w 성수기에는 물건 생산을 하지 못함: 예를 들어, 2021년 8월 15일 피크때, 대체
상품들이 나오지 못함 (매출 저하)
\w 간단히 빨리 만들수 있는 물건들을 만들 것
\w 잘 팔릴 수 있으며 단가가 있으며, 쉽게 만들 수 있는  물건 위주로 만들것
(타르트, 크로아상, 기타)
\w 바쁜 날에는 급하지 않은 것은 하지 않도록 (롤케익, 대판 등); 인력 낭비;
비효율적 인력 운용
\w 
\eit

\subsection{상벌 명확히}
\bit
\w 
\eit

\subsection{지휘체계 확실히}

\section{매장}
\subsection{접객}
\bit
\w 손님이 들어오면 반드시 ``어서오세요!''라고 밝고 친근하게 말할 것
\w 손님이 나갈때는 반드시 ``안녕히 가세요!''라고 밝게 말할 것
\eit

\subsection{전화 응답}
\bit
\w 전화를 받으면 ``안녕하세요. 코롬방제과점입니다.''라고 이야기할 것
\eit

\subsection{트러블시 대처}
\subsubsection{불량제품}

\subsubsection{상한 제품}

\subsubsection{진상손님}

\section{시설}
\subsection{간판}
\bit
\w 오후 5시에 간판 불이 들어왔는지 항상 확인할 것
\w 오후 5시에 창문의 사인물 불을 켤 것
\eit

\section{집기/기계 정리}
\subsection{커피 머신}
\bit
\w 머신 상단에 절대 행주를 말리지 말 것
\w 
\eit


\section{마케팅}
\subsection{블로그 마케팅}
\bit
\w 
\eit

\subsection{SNS 마케팅}
\bit
\w 인스타그램은 센스 있게
\eit

%\vspace*{2cm}

%\tableofcontents
\end{document}



