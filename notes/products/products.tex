\documentclass{myproc}
\usepackage{kotex,pgfplots}
%\usepackage{hangulfontset}
\usepackage{mathptm,mydef}
\usepackage{alltt}
\usepackage[T1]{fontenc}
\usepackage[all,knot]{xy}
\usepackage{MinionPro}
%\usepackage{times}

\usepackage{pgfplots}
\usepackage{tikz}

\usepackage{hyperref}
\hypersetup{
  colorlinks,
  citecolor=black,
  filecolor=black,
  linkcolor=blue,
  urlcolor=black
}

\def\sbf{\sf\bfseries}


\begin{document}

\begin{center}
  \textcolor{blue2}{\large\bf 코롬방제과점 제품/재료}
  \\
  \vspace*{0.8cm}

\end{center}

\small



\section{\textcolor{red2}{시그니처}}
\subsection{\textcolor{blue2}{크림치즈바게트}}
\bit
\w 크림치즈 소스 용기 바꿀 것 (수리 후, 고객에 노출)
\eit

\subsection{\textcolor{blue2}{새우바게트}}
\bit
\w 소스 용기 바꿀 것 (수리 후, 고객에 노출)
\eit

\subsection{\textcolor{blue2}{마늘바게트}}
\bit
\w 소스 용기 바꿀 것 (수리 후, 고객에 노출)
\eit

\subsection{\textcolor{blue2}{주전부리3종}}
\bit
\w 목화솜빵
\w 비파다쿠아즈
\w 맛김새우칩
\eit
\subsection{\textcolor{blue2}{신제품}}
\bit
\w \textcolor{green2}{\bf{}햄치즈 하드롤}: 박미주 개발
\w \textcolor{green2}{\bf{}먹물 누룽지빵}: Brimstone Bread 응용
  \bit
  \w 갈변현상 해결 (pH를 낮추거나 온도를 낮추어서 가공; 버터밀크 사용 또는
  산성제가 들어간 베이킹 파우더 사용)
  \eit
\w \textcolor{green2}{\bf{}Shrimp Po'boy}
\w \textcolor{green2}{\bf{}Lava Bun}
\w \textcolor{green2}{\bf{}Taro Bun}: Taro 수입
\w \textcolor{green2}{\bf{}New Port Dim Sum}: 흑임자속, 모찌, 소보로, 파이
\w \textcolor{green2}{\bf{}단호박빵}: 하루 8개 생산
\eit

\section{\textcolor{red2}{단과자}}
\subsection{\textcolor{blue2}{기본빵}}
\bit
\w 단팥빵: 단팥 직접 쑤는 방안 연구 (taro bun을 위해서는 앙금을 만들어야 함)
\w 소보로빵
\w 우유버터크림빵
\w 완두앙금빵
\w 슈크림빵
\w \textcolor{red2}{R\&D}: 반죽이 노화가 빨라지는 문제 해결:
\textcolor{red2}{유화제}의 사용; 현재 유화제는
합성유화제 (SP) 를 카스테라 만드는 데에만 사용; 대두레시틴 (SolecFS-B) 사용 고려
\eit
\subsection{\textcolor{blue2}{옛날빵}}
\bit
\w 야쿠르트빵
\w 베이비슈
\eit

\subsection{\textcolor{blue2}{특산물빵}}
\bit
\w 해남고구마빵: 양 늘릴 것
\w 흥국쌀빵
\w \textcolor{green2}{\bf{목포삼합 (삼겹살/김치/홍어)}}); 와사비/치즈/단팥
\eit

\section{\textcolor{red2}{건강빵}}
\subsection{\textcolor{blue2}{바게트}}
\bit
\w 퓨라토스 바게트 연구
\w Coloso 학습
\eit

\subsection{\textcolor{blue2}{캄파뉴}}
\bit
\w 비주얼 개선 요망 (덧가루가 많음)
\w 반느통 (발효바구니) 사용 
\eit

\subsection{\textcolor{blue2}{치아바타}}
\bit
\w 껍질에 대한 연구가 필요함
\eit

\subsection{\textcolor{blue2}{포카치아}}

\subsection{\textcolor{blue2}{식빵}}
\bit
\w 들어가는 재료가 무엇인지 확인 (마가린, 가공버터 etc)
\w \textcolor{green2}{\bf{}통밀 식빵}
\w \textcolor{green2}{\bf{}큐브 식빵}: 크림큐브식빵 (coloso)
\eit

\section{\textcolor{red2}{페이스트리}}
\bit
\w T45 밀가루 사용
\w 반죽을 냉동시켜놨다가 자주 구울 것
\w 페이스트리는 포장을 하면 눅눅해짐; 상온에서 팔 것
\eit

\subsection{\textcolor{blue2}{크로아상}}
\bit
\w 플레인, 투톤, 먹물 고르곤졸라, 녹차
\w 반죽을 냉동해 놓고, 조금씩 자주 굽는 방법
\eit
\subsection{\textcolor{blue2}{뺑오쇼콜라}}
\subsection{\textcolor{blue2}{대니쉬}}
\subsection{\textcolor{blue2}{몽블랑}}

\section{\textcolor{red2}{브리오슈}}
\bit
\w 브리오슈 반죽 개선 요망
\eit
\subsection{\textcolor{blue2}{알라스카}}
\subsection{\textcolor{green2}{망고브리오슈}}

\section{\textcolor{red2}{튀김}}
\subsection{\textcolor{blue2}{팥도너츠}}
\subsection{\textcolor{blue2}{고로케}}
\subsection{\textcolor{blue2}{소보루꽈배기}}
\subsection{\textcolor{blue2}{이탈리아노}}

\section{\textcolor{red2}{조리빵}}
\subsection{\textcolor{blue2}{먹물오징어}}
\subsection{\textcolor{blue2}{불고기또띠야}}
\subsection{\textcolor{blue2}{롱소세지}}

\section{\textcolor{red2}{카스텔라}}
\subsection{\textcolor{blue2}{밀봉 카스텔라}}
\bit
\w 포장 연구
\eit

\subsection{\textcolor{blue2}{대왕 카스텔라}}
\bit
\w 생크림 카스텔라 (스타벅스) 벤치마킹
\w 불도장
\eit

\subsection{\textcolor{blue2}{대판 카스텔라}}
\bit
\w 카스텔라 박스가 너무 높음 (재조정)
\eit
\subsection{\textcolor{blue2}{마블 카스텔라}}
\subsection{\textcolor{blue2}{엔젤 쉬폰}}


\section{\textcolor{red2}{샌드위치}}
\subsection{\textcolor{blue2}{에그모닝}}
\subsection{\textcolor{blue2}{닭가슴살 샌드위치}}
\subsection{\textcolor{blue2}{햄치즈 샌드위치}}
\subsection{\textcolor{blue2}{햄버거}}
\subsection{\textcolor{green2}{Panini}}
\subsection{\textcolor{green2}{카프레제 샌드위치}}
\bit
\w 치아바타, 생모짜렐라, 발사믹비네거, 버진 올리브오일, 바질페스토 사용 (유투브 참고)
\eit

\section{\textcolor{red2}{구움과자}}
\subsection{\textcolor{blue2}{머핀}}
\bit
\w \textcolor{green2}{\bf{}Blueberry Muffin}: 미국쪽 레시피 검색 (굵은 설탕
이용); Almond flour를 이용한 머핀 연구
\bit
\w 스타벅스 머핀 벤치마킹
\eit

\w \textcolor{green2}{\bf{}New York-Style Coffee Cake Crumb Muffin}
\w \textcolor{green2}{\bf{}New York-Style Apple Muffin}
\w \textcolor{green2}{\bf{}Apple Oatmeal Muffin}
\eit

\subsection{\textcolor{blue2}{마들렌}}
\bit
\w 더 다양화: 녹차 코팅, etc
\w 모양 수정: 조개 모양으로
\w 포장 연구
\eit

\subsection{\textcolor{blue2}{갈레트}}
\subsection{\textcolor{blue2}{피낭시에}}

\subsection{\textcolor{blue2}{쿠키류}}
\bit
\w 아몬드전병/코코넛전병/참깨전병: ok
\w 사브레코코/판치노이/쇼콜라아망드: 판매 부진; 맛이 평범함
\w 동물쿠키
\w 미국식 쿠키
\eit


\section{\textcolor{red2}{디저트}}
\subsection{\textcolor{blue2}{마카롱}}
\bit
\w 세트 포장 연구
\eit
\subsection{\textcolor{blue2}{다쿠아즈}}
\bit
\w 세트 포장 연구
\eit
\subsection{\textcolor{blue2}{타르트}}
\bit
\w 받침 연구
\w 생산량 늘릴 것
\eit

\subsection{\textcolor{blue2}{크림파이슈}}
\subsection{\textcolor{green2}{에클레어}}

\section{\textcolor{red2}{케익}}
\subsection{\textcolor{blue2}{생크림 케익}}
\bit
\w 파스퇴르 생크림으로 샌딩, 아이싱
\w 작업성이 좋은 컴파운드로 장식
\eit

\subsection{\textcolor{blue2}{버터크림 케익}}
\subsection{\textcolor{blue2}{디자인 케익}}
\bit
\w 레드벨벳 케이크
\eit

\subsection{\textcolor{blue2}{조각케익}}
\bit
\w 포장 연구
\eit

\subsection{\textcolor{blue2}{코코넛케익}}
\subsection{\textcolor{blue2}{기로쉬}}
\subsection{\textcolor{green2}{Pancake Bread (Trader Joe's)}}

\section{\textcolor{red2}{재료}}
\subsection{\textcolor{blue2}{밀가루}}
\bit
\w 단백질 함량 (회분율)에 따른 분류:
  \bit
  \w T45: 10$\sim$12\%: 페이스트리 (크로아상); 브리오슈, 식빵
  \w 박력분: 9\%
  \w T55: 10$\sim$11\%: 케이크/구움과자/치아바타 (T55 Patiserrie), 바게트 (T55
  pour pain courant)
  \w 중력분: 10\%
  \w T65: 10$\sim$11\%: 바게트/하드롤 계열; 정통 바게트는 T65를 사용해야 함
  \w 강력분: 13\%
  \eit
\eit
\subsection{\textcolor{blue2}{버터}}
\bit
\w 유지방 80\% 이상
\w 발효 버터 (sour butter) vs  감성 버터 (sweet butter)
\w 가염 버터 (salted butter) vs 무가염 버터 (unsalted butter)
\w 발효 버터 종류
  \bit
  \w Elle\&Vire (프랑스): Gourmet butter로도 불림
  \w Beurre d'Isigny (AOP vs non-AOP)
  \w President (프레지덩; 프랑스)
  \w Anker (뉴질랜드)
  \w Echire (에쉬레; 프랑스): AOP; 헤이즐넛 향
  \w Arla (덴마크): 매일유업 통해 납품 가능
  \w Lescure (프랑스): AOP; 방목사육한 소
  \eit
\w 감성 버터 종류
  \bit   
  \w 비락버터 (한국)
  \eit
\eit
\subsection{\textcolor{blue2}{가공 버터}}
\bit
\w 유지방 30$\sim$80\%; 가공유지, 무지유고형분 섞어 만듬
\w 가공 버터 종류
  \bit
  \w 롯데우유버터화이트 (한국): 유지방 80\% 
  \w PEF 225 (뉴질랜드): 유지방 79\%
  \eit
  \eit
\subsection{\textcolor{blue2}{마가린}}
\bit
\w 식물성
\eit
\subsection{\textcolor{blue2}{쇼트닝}}
\bit
\w 기름
\eit
  
  
\subsection{\textcolor{blue2}{생크림}}
\bit
\w 유지방 18\% 이상 (케이크나 과일용은 30$\sim$50\% 유지방 함유)
\w 분류:
  half\&half (10$\sim$18\%; 음식; 커피);
  light cream (18$\sim$30\%);
  heavy/whipping cream (30$\sim$38\%);
  double cream (45$\sim$\%);
  clotted cream (55$\sim$\%)
\w 종류:
  \bit
  \w 파스퇴르 생크림 (후레쉬엘 045): 국산유크림 100\% (유지방 45\% 이상)
  \w 매일 생크림: 국산 유크림 100\% (유지방 38\% 이상)
  \w 덴마크 생크림: 국산 유크림 100\% (유지방 36\% 이상)
  \w 서울우유 생크림: 국산 유크림 100\% (유지방 38\% 이상)
  \eit
\w 장점: 맛/풍미가 좋고, 합성첨가물이 들어가지 않음
\w 단점: 비싸고, 유통기한, 작업성에 단점
\eit
\subsection{\textcolor{blue2}{휘핑크림}}
\bit
\w 동물성 휘핑크림: 동물성 생크림에 첨가물 (주로 카라기난) 을 넣은 것 (조지방 35\% 이상)
   \bit
   \w 프레지덩 휘핑크림 (프랑스): 유크림 99.78\%
   \w 레스큐어 휘핑크림 (프랑스): 유크림 99.98\% (조지방 35.1\% 이상)
   \w 알라 휘핑크림 (덴마크)
   \w 매일 휘핑크림: 국산; 유크림 99.6\%
   \w 매일 휘핑크림 35: 독일 OEM; 유크림 98\% (조지방 35\% 이상)
   \w 매일 휘핑크림 골드: 국산/외산 재료 혼합
   \eit
\w 식물성 휘핑크림: 식물성 유지를 이용
   \bit
   \w 롯데푸드 화인휘프5000
   \eit
\w 컴파운드 휘핑크림: 두가지를 혼합
   \bit
   \w 롯데푸드 화인휘프5000
   \eit
\w 카라기난은 한때 발암 논란이 있었던 성분이지만, 이후 연구에 의하면 문제없어보임
\eit

\subsection{\textcolor{blue2}{유화제}}
\bit
\w 화학유화제: S.P.
\w 천연유화제: 레시틴 기반
\eit

\subsection{\textcolor{blue2}{트레할로스}}
\bit
\w 유해 논란이 있으나 해결되지 않았음
\eit

\subsection{\textcolor{green2}{맥아 파우더  (Malt Powder)}}
\bit
\w 몰트 가루로 검색
\w 몰트 엑기스는 단 맛을 첨가하지 않고 바게트의 색을 진하게 하는데 사용가능
\eit
\subsection{\textcolor{green2}{맥아분유 파우더  (Malted Milk Powder)}}
\bit
\w 맥아밀크셰이크/Pancake Bread에 사용
\w 대체품: 몰트가루, 분유, 소금  혼합
\eit
\subsection{\textcolor{green2}{버터밀크 파우더 (Buttermilk Powder)}}
\bit
\w iHerb에서 구입 가능
\eit

\subsection{\textcolor{green2}{Taro}}
\bit
\w 식품 수입 법인 설립
\eit


\vspace*{1cm}

\tableofcontents

\end{document}



